\chapter*{Lời nói đầu}
\addcontentsline{toc}{chapter}{Lời nói đầu}

Các phân tích thống kê những mô hình gắn với dữ liệu có chu kỳ cung cấp cho chúng ta một phép xấp xỉ đủ tốt để mô phỏng các hiện tượng tự nhiên. Ví dụ như mô hình mô hình biến dạng theo chu kỳ có tham số chuyển và hàm hồi quy chưa biết trong bài báo \cite{bercu} có thể dùng để mô phỏng tín hiệu điện tim ở người khỏe mạnh hoặc ở người bị loạn nhịp tim. Tuy nhiên, mô hình trong bài báo năm 2012 này được thiết kế với một biến đầu vào nên cần mở rộng các kết quả đã đạt được lên mô hình nhiều biến và bổ sung các tham số quan trọng khác như chiều cao và co giãn chưa được đề cập, cùng với các ước lượng, sự hội tụ và tính tiệm cận chuẩn của chúng. Từ những lý do trên, khóa luận sẽ tập trung nghiên cứu các mô hình biến dạng nhiều biến có tham số chuyển, tham số chiều cao, co giãn và hàm hồi quy chưa biết trước.

Các mô hình biến dạng theo chu kỳ là các mô hình hồi quy bán tham số với một hàm tuần hoàn chưa biết. Xét các tập dữ liệu có liên quan đến hàm tuần hoàn và khác nhau ở ba tham số, một tham số chuyển, một tham số chiều cao và một tham số co giãn. Cụ thể mô hình trên được biểu diễn như sau
\begin{align}
    Y_{i, j}=a_{j} f\left(X_{i}-\theta_{j}\right)+v_{j}+\varepsilon_{i, j},
    \label{1.1}
\end{align}
trong đó $1 \leq j \leq p$ và $1 \leq i \leq n$, hàm dạng $f$ có chu kỳ và các $X_{i}$ là biến ngẫu nhiên, độc lập cùng phân phối.

Với $p=1, a_{1}=1$ và $v_{1}=0$, Bercu và Fraysse \cite{bercu} đề xuất một phương pháp đệ quy để ước lượng tham số chuyển $\theta_{1}$. Trong khóa luận này, tôi xin trình bày chi tiết các mở rộng của Fraysse [1] nhằm ước lượng tham số chiều cao $v$, tham số chuyển $\theta$ và tham số co giãn với số chiều $p$ tùy ý cho bởi
\begin{align}
    v=\left(\begin{array}{c}
v_{1} \\
\vdots \\
v_{p}
\end{array}\right), \quad \theta=\left(\begin{array}{c}
\theta_{1} \\
\vdots \\
\theta_{p}
\end{array}\right), \quad a=\left(\begin{array}{c}
a_{1} \\
\vdots \\
a_{p}
\end{array}\right).
\label{1.2}
\end{align}

Đầu tiên ta sẽ ước lượng vector tham số chiều cao $v$ bằng cách thiết lập một ước lượng vững tự nhiên $\widehat{v}_n$ hội tụ về $v$ và thiết lập tính tiệm cận chuẩn dựa trên định lý giá trị trung tâm cho martingales.

Sau đó, ta tiếp tục ước lượng tham số chuyển $\theta$ bằng thuật toán Robbins - Monro nhiều chiều với giả định rằng tồn tại một hàm $\phi: \mathbb{R}^{p} \rightarrow \mathbb{R}^{p}$, không chịu ảnh hưởng bởi $\theta$, sao cho $\phi(\theta)=0$. Khi đó 
\begin{align}
    \widehat{\theta}_{n+1}=\widehat{\theta}_{n}+\gamma_{n} T_{n+1},
    \label{1.3}
\end{align}
trong đó $\left(\gamma_{n}\right)$ là một dãy các số thực dương giảm dần về 0 và dãy $\left(T_{n}\right)$ là dãy các vector ngẫu nhiên sao cho $\mathbb{E}\left[T_{n+1} \mid \mathcal{F}_{n}\right]=\phi\left(\widehat{\theta}_{n}\right)$ trong đó $\mathcal{F}_{n}$ là $\sigma$-đại số các biến cố đã xảy ra đến thời điểm thứ $n$. Dưới các điều kiện chuẩn trên hàm $\phi$ và trên dãy $\left(\gamma_{n}\right)$, một kết quả quan trọng trong \cite{bercu} là $\widehat{\theta}_{n}$ tiến về $\theta$ hầu chắc chắn. Tính tiệm cận chuẩn của $\widehat{\theta}_{n}$ có thể được tìm thấy trong \cite{pelletier_weak_convergence} trong khi luật mạnh dạng toàn phương và luật loga-lặp được thiết lập trong \cite{pelletier_on_the_almost}.\\
Tiếp đó, ta ước lượng tham số co giãn $a$ qua một ước lượng vững mạnh sử dụng ước lượng tiên nghiệm $\theta$. Cuối cùng là chứng minh chặt chẽ cách sử dụng ước lượng đệ quy Nadaraya-Watson có trọng \cite{duflo} nhằm ước lượng hàm hồi quy $f$. Việc này có sử dụng thêm các ước lượng trước đó cho $v, \theta$ và $a$, tương ứng với $\widehat{v}_{n}$, $\widehat{\theta}_{n}$ và $\widehat{a}_{n}$. Với mọi $x \in \mathbb{R}$, ước lượng của hàm $f$ xác định bởi
\begin{align}
    \widehat{f}_{n}(x)=\sum_{j=1}^{p} \omega_{j}(x) \widehat{f}_{n, j}(x)
    \label{1.4}
\end{align}
với 
$$
\widehat{f}_{n, j}(x)=\frac{1}{\widehat{a}_{n, j}} \frac{\sum_{i=1}^{n} W_{i, j}(x)\left(Y_{i, j}-\widehat{v}_{i-1, j}\right)}{\sum_{i=1}^{n} W_{i, j}(x)}
$$
và
$$
W_{n, j}(x)=\frac{1}{h_{n}} K\left(\frac{X_{n}-\widehat{\theta}_{n-1, j}-x}{h_{n}}\right),
$$
trong đó $\widehat{v}_{n, j}, \widehat{\theta}_{n-1, j}$ và $\widehat{a}_{n, j}$ tương ứng với thành phần thứ $j$ của $\widehat{v}_{n}, \widehat{\theta}_{n-1}$ và $\widehat{a}_{n}$. Hơn nữa, hạt nhân $K$ là một hàm mật độ được chọn trước và băng tần $\left(h_{n}\right)$ là một dãy các số thực dương giảm về 0.

Toàn bộ khóa luận cấu trúc thành 4 chương. Chương 1 trình bày mô hình và các giả thuyết cần thiết để thực hiện quá trình phân tích thống kê. Chương 2 là các kiến thức chuẩn bị trước khi đi vào chứng minh ở chương 3 dành cho việc ước lượng tham số của vector $v$, thuật toán Robbins-Monro cho việc các ước lượng tham số  $\theta$ và ước lượng tham số cho vector tham số co giãn $a$. Trong ba nội dung này, khóa luận trình bày chi tiết các chứng minh để thiết lập sự hội tụ hầu chắc chắn của $\widehat{v}_{n}, \widehat{\theta}_{n}$ và $\widehat{a}_{n}$ cùng với tính tiệm cận chuẩn của chúng. Luật mạnh dạng toàn phương cũng được cung cấp cho ba ước lượng này. Ngoài ra, phần cuối của chương 3 giải quyết vấn đề ước lượng phi tham số của hàm $f$. Dưới các giả định tiêu chuẩn thông thường trên hạt nhân $K$, ta sẽ chứng minh hội tụ từng điểm hầu chắc chắn của $\widehat{f}_{n}$ về hàm $f$ và tính tiệm cận chuẩn của $\widehat{f}_{n}$. Hơn nữa, các ràng buộc cho sự xác định duy nhất liên quan đến mô hình. Chương 4 bao gồm các thí nghiệm trên số liệu mô phỏng bằng ngôn ngữ lập trình R, minh họa cho hiệu năng của quy trình ước lượng bán tham số. Cuối cùng, là các kết luận, định hướng nghiên cứu trong tương lai, phụ lục và tài liệu tham khảo cho khóa luận.