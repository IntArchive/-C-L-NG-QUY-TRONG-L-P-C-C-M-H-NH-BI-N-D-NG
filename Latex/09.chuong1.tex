
\chapter{Giới thiệu mô hình và các giả thiết}
Ta sẽ tập trung vào mô hình, với \(1 \leq j \leq p\), trong đó \(p \geq 2\), và \(1 \leq i \leq n\)

\[ Y_{i, j}=a_{j} f\left(X_{i}-\theta_{j}\right)+v_{j}+\varepsilon_{i, j} , \]
trong đó, với mọi \(1 \leq j \leq p, a_{j} \neq 0\). Với mọi \(1 \leq i \leq n\), nhiễu \(\left(\varepsilon_{i, j}\right)\) là một dãy các biến ngẫu nhiên độc lập với giá trị trung bình bằng không và phương sai \(\mathbb{E}\left[\varepsilon_{i, j}^{2}\right]=\sigma_{j}^{2}\), và độc lập với các biến ngẫu nhiên \(X_{i}, i=\overline{1,n}\),. Ngoài ra, như trong \cite{bercu}, ta cần các giả thiết sau.

$\left(\mathcal{H}_{1}\right)$ Các quan trắc $\left(X_{i}\right)$ là độc lập, cùng phân phối với hàm mật độ $g$, dương trên giá của nó $[-1 / 2 ; 1 / 2]$. Hơn nữa, $g$ liên tục trên $\mathbb{R}$, khả vi cấp 2 với đạo hàm bị chặn.

$\left(\mathcal{H}_{2}\right)$ Hàm dạng \(f\) là hàm đối xứng, bị chặn, tuần hoàn với chu kỳ 1.

Ta tạm thời chấp nhận rằng mô hình biến dạng theo chu kỳ \ref{1.1} có thể xác định được với cách lựa chọn giả thuyết $\left(\mathcal{H}_{3}\right)$ và $\left(\mathcal{H}_{3}\right)$ sau đây. 

$
\begin{aligned}
& \left(\mathcal{H}_{3}\right) \quad \int_{-1 / 2}^{1 / 2} f(x) d x=0 , \\
& \left(\mathcal{H}_{4}\right) \quad a_{1}=1, \theta_{1}=0 \text { và } \max _{1 \leq j \leq p}\left|\theta_{j}\right|<1 / 4 .
\end{aligned}
$

Nói theo cách khác là tồn tại duy nhất ước lượng các tham số \(a, \theta, v\) cũng như là hàm dạng \(f\) với điều kiện $\left(\mathcal{H}_{3}\right)$ và $\left(\mathcal{H}_{4}\right)$ đúng. Điều này sẽ được làm rõ trong phần cuối cùng của chương 4.


