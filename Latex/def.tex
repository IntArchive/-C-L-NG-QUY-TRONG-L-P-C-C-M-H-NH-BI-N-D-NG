%%-----Định nghĩa lại tên tiếng việt ----
\renewcommand\today{Ngày \number\day{} tháng \number\month{} năm \number\year{}}
\newcommand\stoday{\number\day/\number\month/\number\year{}}
\providecommand*{\contensname}{}\renewcommand*{\contentsname}{Mục lục}
\providecommand*{\listfigurename}{}\renewcommand*{\listfigurename}{Danh sách các hình}
\providecommand*{\listtablename}{}\renewcommand*{\listtablename}{Danh sách các bảng}
\providecommand*{\indexname}{}\renewcommand*{\indexname}{Chỉ mục}
\providecommand*{\figurename}{}\renewcommand*{\figurename}{Hình}
\providecommand*{\tablename}{}\renewcommand*{\tablename}{Bảng}
\providecommand*{\partname}{}\renewcommand*{\partname}{Phần}
\providecommand*{\appendixname}{}\renewcommand*{\appendixname}{Phụ lục}
\providecommand*{\chaptername}{}\renewcommand*{\chaptername}{Chương}
\providecommand*{\bibname}{}\renewcommand*{\bibname}{Tài liệu tham khảo}
\providecommand*{\abstractname}{}\renewcommand*{\abstractname}{Tóm tắt}
\providecommand*{\refname}{}\renewcommand*{\refname}{Tài liệu dẫn}
\providecommand*{\pagename}{}\renewcommand*{\pagename}{Trang}
\providecommand*{\proofname}{}\renewcommand*{\proofname}{{\bf Chứng minh}}
\def\up{\MakeUppercase}
%%----------Ký hiệu chuẩn cho tập hợp số tự nhiên, số nguyên, số hữu tỷ, số thực và số ph 
\def\N{\mathbb{N}}
\def\Z{\mathbb{Z}} 
\def\Q{\mathbb{Q}}
\def\R{\mathbb{R}}
\def\C{\mathbb{C}}
%%----------Định nghĩa lại cho dấu tương đương và dấu suy ra.
\def\iffs{\Leftrightarrow} \def\suy{\Rightarrow} 
%---------------------------------------------------------
\newcommand{\id}[1]{{\em #1}\index{#1}} %Tạo từ chỉ mục và in nghiêng từ cần làm chỉ mục
%%---------------------------------------------------
%%------------Định nghĩa lại môi trường Định lý, Định nghĩa, Mệnh đề, ...
\newtheorem{dl}{Định lý}[section]
\newtheorem{md}[dl]{Mệnh đề}
\newtheorem{hq}[dl]{Hệ quả}
\newtheorem{bd}[dl]{Bổ đề}
\newtheorem{gt}[dl]{Giả thuyết}
\newtheorem{tc}[dl]{Tính chất}
\theoremstyle{definition}\newtheorem{dn}[dl]{Định nghĩa}
\theoremstyle{definition}\newtheorem{cy}[dl]{Chú ý}
\theoremstyle{definition}\newtheorem{nx}[dl]{Nhận xét}
%%----- Tạo môi trường có tên Ví dụ và đánh số cho ví dụ (cùng cấp với section)----- 
\newcounter{danhsovidu}[section] %tao o dem moi de danh so cho vi du
\newenvironment{vidu}{\stepcounter{danhsovidu}\par\noindent{\bfseries Ví dụ \thesection.\thedanhsovidu.}}{\par}
%%---------------------------------------------------
%%------------khung cho phần chứng minh thêm vào
% Define the new command \bscm
\newcommand{\bscm}[1]{% 
\begin{tcolorbox}[colback=red!5!white,colframe=red!75!black]
#1
\end{tcolorbox}
}
%%---------------------------------------------------
%%------------ text in math mode
% Define the new command \tim :: text in mathmode
\newcommand{\tim}[1]{\left(\text{#1}\right)}
\newcommand{\tref}[1]{$\left(\ref{#1}\right)$}



%%---------------------------------
%%
%%----------Tự tạo header và footer riêng-----------------------%%
\makeatletter
\newcommand{\ps@myplain}{%khai báo kiểu định dạng mới myplain
\renewcommand{\@oddhead}{\textit{Khóa luận tốt nghiệp - Chuyên ngành Xác suất -Thống kê}\hrulefill Trang \thepage} 
%tạo header trang lẻ
\renewcommand{\@evenhead}{Trang \thepage \hrulefill \textit{Khóa luận tốt nghiệp cử nhân sư phạm toán học}} 
%tạo header trang chẵn
 \renewcommand{\@oddfoot}{\mbox{\emph{Đinh Minh Hải - Cử nhân Sư phạm Toán học K46}}\dotfill Trang \thepage} 
% tạo footer trang lẻ
\renewcommand{\@evenfoot}{\@oddfoot}} 
% tạo footer trang chẵn giống footer trang lẻ
\makeatother
%%----------------End of file def.tex
