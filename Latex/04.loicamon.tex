\begin{center}
\LARGE{\textbf{Lời cảm ơn}}
\end{center}
Em xin bày tỏ lòng biết ơn sâu sắc đến thầy Nguyễn Phát Đạt, người đã hướng dẫn tận tâm giúp em hoàn thành khóa luận này. Sự nhiệt tình của thầy luôn tạo động lực cho em vượt qua mọi khó khăn trong nghiên cứu và thực hiện khóa luận. Xin cảm ơn triết lý học võ tự vệ của người làm xác suất mà thầy có dịp thảo luận với sinh viên, nó thực sự rất hay.

Em cũng xin gửi lời cảm ơn chân thành đến các thầy cô trong tổ Toán Ứng dụng đã tạo điều kiện và hỗ trợ em trong suốt quá trình nghiên cứu và thực hiện khóa luận.

Xin cảm ơn thầy Đào Huy Cường vì một học phần "Xác suất thống kê" vô cùng hay và ý nghĩa, đã đưa em từng bước từng bước, xây dựng nên tảng kiến thức cho các học phần khác về xác suất sau này mà cuối cùng là khóa luận tốt nghiệp.

Xin cảm ơn thầy Nguyễn Thành Nhân cho những triết lý học tập vô cùng sâu sắc mà thầy truyền đạt trong học phần "Giải tích hàm nhiều biến". Em luôn ghi nhớ những câu chuyện và những hướng dẫn của thầy về cách học, cách chuẩn bị cho các kỳ thi và chuẩn bị cho tương lai dài hạn từ một thời điểm rất lâu trước đó. Đặc biệt, em sẽ luôn ghi nhớ tháp tam giác phân loại những người làm toán và sự khiêm tốn của thầy khi nói về chỗ mà thầy đứng trong tháp tam giác đó. 

Xin cảm ơn thầy Phạm Duy Khánh cho những tiết học chặt chẽ về toán ở học phần "Lý thuyết tối ưu" và sự nghiêm túc của thầy trong học tập và làm việc nhưng vô cùng thoải mái ngoài đời. Em quý thầy nhưng cũng rất rén khi gặp thầy.

Xin cảm ơn bố mẹ tôi, bố Thêm, mẹ Hường đã nuôi dưỡng tôi nên người và lo lắng cho tôi trong suốt năm tháng học tập ở trường Đại học Sư phạm. 

Xin cảm ơn anh trai tôi - Minh Hoàng vì sự có mặt của anh trong suốt thời thơ ấu và trong nền giáo dục gian truân mà khắc khổ từ gia đình dành cho chúng tôi. Mong anh có thể đi những bước vững chãi trong cuộc sống của anh.

Xin cảm ơn vợ tôi - Gia Hân đã luôn động viên tinh thần và tiếp tế nguồn lực cho tôi trong suốt những ngày làm khóa luận khó khăn. Cảm ơn em cho những góp ý sâu sắc về những điều cần phải làm trong các vấn đề bên lề của khóa luận này. Cảm ơn em vì phần đồ ăn chuẩn bị cho năm ngày làm việc liên tục để hoàn thành khóa luận này.

Xin cảm ơn người bạn lâu năm - Minh Đức đã đánh cờ giải lao với tôi sau những lần chứng minh định lý, và những giờ bơi đua rất tốn thể lực nhưng đầy sảng khoái mỗi tuần.

Xin cảm ơn Tấn Phát và Giáp Thân đã chinh chiến cùng tôi trên sân chơi Olympic Tin học sinh viên. Thời gian đó thực sự là một bước ngoặt về chất để tôi chuẩn bị cho sự mài dũa tư duy cá nhân, tư duy học toán thông qua việc viết, trình bày và giải rất nhiều ra nháp, đồng thời là cả tư duy về lập trình thuật toán - điều giúp ích cho tôi rất nhiều khi viết những dòng code để ước lượng hàm hồi quy.

Xin cảm ơn Uyên Phương vì những hỗ trợ của Phương trong kỳ Thực tập Sư phạm 2 và những chia sẻ của Phương về tinh thần học tập và làm việc nghiêm túc. Xin cảm ơn tinh thần lạc quan và hoan hỉ Phương trong mọi khó khăn của công việc.

Xin cảm ơn Như Ý, Anh Tuấn vì những thiện ý khi nhóm chúng ta làm việc cùng nhau.

Xin cảm ơn anh Quân vì những câu chuyện bên lề khóa luận dở khóc dở cười.

Xin cảm ơn các thầy, các cô ở trường Đại học Sư phạm mà tôi may mắn có dịp học cùng. Các thầy các cô là tấm gương rực rỡ để chúng em học hỏi và phát triển sự nghiệp toán học của riêng mình. Xin cảm ơn các thầy cô rất nhiều.

Lời cuối cùng, xin cảm ơn những đọc giả đã hoan hỉ cùng tôi bước qua rất nhiều những lời cảm ơn bên trên. Tôi không mong các bạn thấy mệt mỏi vì nó thể hiện quá trình học ở trường Sư Phạm đầy ý nghĩa và nhiều màu sắc của tôi ở đây. Và một lý do khác nữa là chúng ta cần lên dây cót tinh thần để bước vào nội dung chính của tài liệu này: khóa luận tốt nghiệp của tôi - Đinh Minh Hải.

\begin{flushright}
{\it Tp.HCM, ngày 03 tháng 05 năm 2023}

Tác giả\hskip 2cm\quad

\vskip 2cm

{\bf Đinh Minh Hải} $\;\;\;\;\;\;\;$ \hskip 1cm 
 \end{flushright}
\thispagestyle{empty}
