\chapter*{Kết luận}
Trong khóa luận này, tôi đã trình bày chứng minh các kết quả liên quan đến mô hình biến dạng có chu kỳ \tref{1.1} bao gồm sự hội tụ của các ước lượng tham số và phi tham số, còn có các kết quả về tính tiệm cận chuẩn, luật loga-lặp và luật mạnh dạng toàn phương. Việc ứng dụng các kết quả đó để xây dựng thuật toán và viết chương trình chạy trên dữ liệu mô phỏng đem đến những kết quả tốt, chứng tỏ phương pháp thực sự hiệu quả trên dữ liệu mô phỏng, đồng thời các biểu hiện tích cực đó của phương pháp mở ra những hy vọng về sự hiệu quả khi ứng dụng phương pháp trên cho các dữ liệu thực tế.

Các công việc tiếp theo cần phải làm là xây dựng thêm các lý thuyết để loại bỏ việc sử dụng dấu của $a$ và hệ số Fourier thứ nhất của hàm $f$, làm nền tảng để ứng dụng phương pháp trên cho các dữ liệu tuần hoàn trong thực tế mà tiêu biểu là dữ liệu điện tim. Đó cũng chính là các nội dung tôi mong muốn làm được sau khi kết thúc khóa luận.

Qua việc thực hiện khóa luận, tôi đã học thêm được rất nhiều kiến thức về lĩnh vực "Xác suất phi tuyến" như là các định lý "Luật số lớn", "Định lý giới hạn trung tâm" cho những đối tượng toán học khác nhau. Một số lý thuyết đời thường mà thú vị về trò chơi công bằng - xuất phát điểm của lý thuyết martingale,... . Mặc dù, việc nghiên cứu nhiều lần làm tôi mệt, nhưng hành trình này cũng mang đến sự vui vẻ trong tâm thức mà qua đó dìu dắt tôi hoàn thành trọn vẹn khóa luận này. Tôi không đặt kỳ vọng các bạn sinh viên Toán lứa sau sẽ làm khóa luận, song vẫn động viên các bạn lăn xả nếu có thể. Vì lý do gì nào? Tôi hy vọng các bạn có thể tự tìm ra cho mình.

Một lần nữa, xin gửi lời cảm ơn chân thành đến thầy Nguyễn Phát Đạt, vì đã cho phép và hỗ trợ tôi từ lúc bắt đầu đến khi kết thúc khóa luận này. Cảm ơn sự gần gũi như một người anh của thầy, những sự động viên của thầy đã giúp em dũng cảm hơn rất nhiều để đối mặt với những khó khăn tri thức khi thực hiện đề tài này. Cảm ơn thầy... rất nhiều.