\chapter{Kiến thức chuẩn bị}
%%%%%%%%%%%%%%%%%%%%%%%%%%%%%%%%%%%%%%%%%%%%%%%%%%%%%%%%%%%%%%%%%%
%%%%%%%%%%%%%% Kỳ vọng có điều kiện
%%%%%%%%%%%%%%%%%%%%%%%%%%%%%%%%%%%%%%%%%%5%%%%%%%%%%%%%%%%%%%%%%%
\section{Kỳ vọng có điều kiện và Martingale}
\subsection{Kỳ vọng có điều kiện}
{\dn Cho $\mathbf{X}: \Omega \rightarrow \mathbb{R}^{n}$ là một vector ngẫu nhiên được định nghĩa trên không gian xác suất $(\Omega, \mathcal{A}, \mathbb{P})$. Ánh xạ chiếu $\pi_{i}: \mathbb{R}^{n} \rightarrow \mathbb{R}$ của vector $n$ chiều $X$ lên thành phần thứ $i$ của nó, $X_{i}=\pi_{i}(\mathbf{X})$, là một hàm đo được Borel. Dẫn đến $X_{i}=\pi_{i} \circ \mathbf{X}$ là một biến ngẫu nhiên đối với mỗi $i \in \overline{1,n}$. Hơn nữa, không gian biến cố được sinh bởi vector ngẫu nhiên $\mathbf{X}$ được cho bởi
$$
\sigma(\mathbf{X})=\sigma\left(X_{1}, \ldots, X_{n}\right) .
$$}
\subsubsection{Kỳ vọng có điều kiện của biến ngẫu nhiên}
\quad Giả sử $\mathcal{F}$ là $\sigma$-đại số con của $\sigma$-đại  $\mathcal{A}$.
    Kỳ vọng có điều kiện của biến ngẫu nhiên $X \geq 0$ đối với $\mathcal{F}$ là biến ngẫu nhiên suy rộng không âm
\[
\mathbb{E}(X \mid \mathcal{F}): \Omega \rightarrow[0, \infty]
\]
sao cho

(i) $\mathbb{E}(X \mid \mathcal{F})$ là $\mathcal{F}$-đo được,

(ii) với mọi $A \in \mathcal{F}$
\[
\int_A X \, d\mathbb{P} = \int_A \mathbb{E}(X \mid \mathcal{F}) \, d\mathbb{P} .
\]

Giả sử $X$ là biến ngẫu nhiên bất kỳ sao cho với xác suất một
\[
\min \left\{\mathbb{E}\left(X^{+} \mid \mathcal{F}\right), \mathbb{E}\left(X^{-} \mid \mathcal{F}\right)\right\}<\infty .
\]

Khi đó ta nói: $X$ có kỳ vọng có điều kiện đối với $\sigma$-trường $\mathcal{F}$, và gọi
\[
\mathbb{E}(X \mid \mathcal{F})=\mathbb{E}\left(X^{+} \mid \mathcal{F}\right)-\mathbb{E}\left(X^{-} \mid \mathcal{F}\right)
\]
là kỳ vọng có điều kiện của $X$ đối với $\mathcal{F}$.

Nếu $Y$ là biến ngẫu nhiên và $\sigma(Y)$ là $\sigma$-trường $Y^{-1}(\mathcal{B}(\mathbb{R}))$ thì ta viết
$$
\mathbb{E}(X \mid Y):=\mathbb{E}(X \mid \sigma(Y)).
$$

\subsubsection{Kỳ vọng có điều kiện cho trước vector ngẫu nhiên}
Xét biến ngẫu nhiên $X: \Omega \rightarrow \mathbb{R}$ sao cho $\mathbb{E}|X|<\infty$ và vector ngẫu nhiên 
$\mathbf{Y}: \Omega \rightarrow \mathbb{R}^{n}, $ $n \in \mathbb{N}$ được định nghĩa trên cùng một không gian xác suất $(\Omega, \mathcal{F}, \mathbb{P})$. Kỳ vọng có điều kiện của biến ngẫu nhiên $X$ cho trước vector ngẫu nhiên $\mathbf{Y}$ được định nghĩa là $\mathbb{E}[X \mid \mathbf{Y}] \triangleq \mathbb{E}[X \mid \sigma(\mathbf{Y})]$.
%%%%%%%%%%%%%%%%%%%%%%%%%%%%%%%%%%%%%%%%%%%%%%%%%%%%%%%%%%%%%%%%%%
%%%%%%%%%%%%%% Các tính chất của kỳ vọng có điều kiện
%%%%%%%%%%%%%%%%%%%%%%%%%%%%%%%%%%%%%%%%%%5%%%%%%%%%%%%%%%%%%%%%%%
\subsubsection{Các tính chất của kỳ vọng có điều kiện}

Ta quy ước rằng khi viết kỳ vọng có điều kiện thì ta ngầm giả sử kỳ vọng có điều kiện tồn tại.
Đồng thời, ta đồng nhất các biến ngẫu nhiên bằng nhau hầu chắc chắn. Các đẳng thức, bất đẳng thức hiểu theo nghĩa: được thực hiện hầu chắc chắn.
\begin{enumerate}
    \item Nếu $X$ là $\mathcal{F}$-đo được thì $\mathbb{E}(X \mid \mathcal{F})=X$. Đặc biệt, nếu $C$ là hằng số thì $\mathbb{E}(C \mid \mathcal{F})=C$.
    \item Nếu $X \leq Y$ thì $\mathbb{E}(X \mid \mathcal{F}) \leq \mathbb{E}(Y \mid \mathcal{F})$. Đặc biệt, ta có bất đẳng thức
    $$
    |\mathbb{E}(X \mid \mathcal{F})| \leq \mathbb{E}(|X| \mid \mathcal{F}).
    $$
    \item Nếu $a, b \in \mathbb{R}$ thì $\mathbb{E}((a X+b Y) \mid \mathcal{F})=a \mathbb{E}(X \mid \mathcal{F})+b \mathbb{E}(Y \mid \mathcal{F})$. \\
    \item $\mathbb{E}[\mathbb{E}(X \mid \mathcal{F})]=\mathbb{E} X$. \\
    \item Nếu $\sigma(X)$ và $\mathcal{F}$ độc lập thì $\mathbb{E}(X \mid \mathcal{F})=\mathbb{E} X$. Đặc biệt, nếu $X, Y$ độc lập thì $\mathbb{E}(X \mid Y)=\mathbb{E} X$.
    \item Nếu $\mathcal{F}_1 \subset \mathcal{F}_2 \subset $ thì
    $$
    \mathbb{E}\left[\mathbb{E}\left(X \mid \mathcal{F}_2\right) \mid \mathcal{F}_1\right]=\mathbb{E}\left[\mathbb{E}\left(X \mid \mathcal{F}_1\right) \mid \mathcal{F}_2\right]=\mathbb{E}\left(X \mid \mathcal{F}_1\right) .
    $$
    \item Nếu $Y$ là $\mathcal{F}$-đo được thì
    $$
    \mathbb{E}(X Y \mid \mathcal{F})=Y \mathbb{E}(X \mid \mathcal{F}).
    $$
    \item Nếu $\mathcal{F}_0=\{\emptyset, \Omega\}$ ($\sigma$-trường tầm thường) thì
    $$
    \mathbb{E}\left(X \mid \mathcal{F}_0\right)=\mathbb{E} X .
    $$
\end{enumerate}
%%%%%%%%%%%%%%%%%%%%%%%%%%%%%%%%%%%%%%%%%%%%%%%%%%%%%%%%%%%%%%%%%%
%%%%%%%%%%%%%% Martingale
%%%%%%%%%%%%%%%%%%%%%%%%%%%%%%%%%%%%%%%%%%5%%%%%%%%%%%%%%%%%%%%%%%
\subsection{Martingale và một số tính chất}
{\dn (Martingale, tham khảo \cite{duflo}) Giả sử rằng $X=\left(X_n\right)$ là dãy các biến ngẫu nhiên khả tích và tương thích với bộ lọc $\mathbb{F} = \left(\mathcal{F}_n\right)$. $X$ được gọi là:
\begin{itemize}
    \item Martingale nếu với mọi số tự nhiên $n$
$$
\mathbb{E}\left[X_{n+1} \mid \mathcal{F}_n\right]=X_n \text { h.c.c . }
$$
    \item Martingale dưới nếu với mọi số tự nhiên $n$
$$
\mathbb{E}\left[X_{n+1} \mid \mathcal{F}_n\right] \geq X_n \text { h.c.c . }
$$
    \item Martingale trên nếu với mọi số tự nhiên $n$
$$
\mathbb{E}\left[X_{n+1} \mid \mathcal{F}_n\right] \leq X_n \text { h.c.c . }
$$
\end{itemize}
}
{\dn (Tham khảo \cite{duflo}) Biến phân bình phương (hoặc đặc trưng bình phương) dự báo được gắn liền với martingale $\left(X_{n}\right)$ với $\langle X\rangle_{0}=0$ và với mọi $n \geq 1$, được định nghĩa 
$$
\langle X\rangle_{n}=\sum_{k=1}^{n} \mathbb{E}\left[\left(\Delta X_{k}\right)^{2} \mid \mathcal{F}_{k-1}\right],
$$
trong đó $\Delta X_{k}=X_{k}-X_{k-1}$ .\\
Ta ký hiệu
$$
\langle X\rangle_{\infty}=\lim _{n \rightarrow \infty}\langle X\rangle_{n} . 
$$}
%%%%%%%%%%%%%%%%%%%%%%%%%%%%%%%%%%%%%%%%%%%%%%%%%%%%%%%%%%%%%%%%%%
%%%%%%%%%%%%%% Martingale
%%%%%%%%%%%%%%%%%%%%%%%%%%%%%%%%%%%%%%%%%%5%%%%%%%%%%%%%%%%%%%%%%%
\subsubsection{Định lý giới hạn trung tâm cho martingale}
{\dl (Định lý giới hạn trung tâm cho martingale, tham khảo [3]) Cho $\left(X_n\right)$ là một martingale bình phương khả tích và $\left(a_n\right)$ là dãy các số thực dương tăng dến vô cùng. Giả sử rằng:
\begin{itemize}
    \item Tồn tại một giới hạn tất định $l>0$ sao cho
    $$
    \frac{\langle X\rangle_n}{a_n} \xrightarrow{\mathcal{P}} l .
    $$
    \item Điều kiện Lindeberg
    Với mọi $\varepsilon>0$,
    $$
    \frac{1}{a_n} \sum_{k=1}^n \mathbb{E}\left[\left|\Delta X_k\right|^2 \mathbb{I}_{\left\{\left|\Delta X_k \geq\right| \varepsilon \sqrt{a_n}\right\}} \mid \mathcal{F}_{k-1}\right] \xrightarrow{\mathcal{P}} 0 .
    $$
\end{itemize}

Khi đó
$$
\frac{1}{\sqrt{a_n}} X_n \xrightarrow{\mathcal{L}} \mathcal{N}(0, l) .
$$

Hơn nữa, khi $l>0$, ta được
$$
\sqrt{a_n}\left(\frac{X_n}{\langle X\rangle_n}\right) \stackrel{\mathcal{L}}{\rightarrow} \mathcal{N}\left(0, l^{-1}\right) .
$$}
\subsubsection{Luật mạnh số lớn cho martingale}
{\dl (Luật mạnh số lớn cho martingale, tham khảo \cite{duflo}) Giả sử $\left(M_n\right)$ là martingale bình phương khả tích với quy trình tăng $\langle M\rangle_n$ và đặt $\langle M\rangle_{\infty}=\lim \langle M\rangle_n$. Khi đó $M_n /\langle M\rangle_n \xrightarrow{\text { h.c.c . }} 0$ trên $\left\{\langle M\rangle_{\infty}=\infty\right\}$. Chính xác hơn, với mọi $\gamma>0$, ta có
$$
M_n /\langle M\rangle_n=\mathrm{o}\left(\left(\left(\ln \langle M\rangle_n\right)^{1+\gamma} /\langle M\rangle_n\right)^{1 / 2}\right) \text{       h.c.c .}
$$
}
\subsubsection{Khái niệm vector martingale}
{\dn (Định nghĩa khái niệm vector martingale bình phương khả tích, tham khảo \cite{duflo}) Trên không gian xác suất $(\Omega, \mathcal{A}, \mathbb{P})$ và $\mathbb{F}=\left(\mathcal{F}_n\right)$ là một bộ lọc. Giả sử $M=\left(M_n\right)$ là một dãy các vector ngẫu nhiên có giá trị trong $\mathbb{R}^d$ và tương thích với bộ lọc $\mathbb{F}$.
\begin{itemize}
    \item $M$ là một martingale bình phương khả tích nếu đối với mọi $n$
$$
E\left[\left\|M_n\right\|^2\right]<\infty \text { và } E\left[M_{n+1}-M_n \mid \mathcal{F}_n\right]=0
$$
\item Biến phân bình phương dự báo được của $M$ là một dãy ngẫu nhiên $\langle M\rangle=\left(\langle M\rangle_n\right)$ các ma trận đối xứng, nửa xác định dương cỡ $d \times d$ được định nghĩa bằng cách đặt $\langle M\rangle_0=0$ và
$$
\begin{aligned}
\langle M\rangle_n-\langle M\rangle_{n-1} & =E\left[\left(M_n-M_{n-1}\right)\left(M_n-M_{n-1}\right)^{\top} \mid \mathcal{F}_{n-1}\right] \\
& =E\left[M_nM_n^{\top} \mid \mathcal{F}_{n-1}\right]-M_{n-1}M_{n-1}^{\top} .
\end{aligned}
$$
\end{itemize}
}
\subsubsection{Định lý giới hạn trung tâm cho vector martingale thực}
{\dl (Định lý giới hạn trung tâm cho vector martingale thực, hệ quả 2.1.10 trong \cite{duflo}) Cho $M$ là một vector martingale bình phương khả tích thực, đáp ứng bộ lọc $\mathbb{F} = \left(\mathcal{F}_n\right)$, có biến phân bình phương dự báo được ký hiệu bởi $\langle M\rangle$. Giả sử rằng, với một dãy thực xác định, ($a_n$) tăng đến $+\infty$, có thêm hai giả định sau:
\begin{itemize}
    \item $a_n^{-1}\langle M\rangle_n \xrightarrow{\mathrm{P}} \Gamma$.
    \item Điều kiện Lindeberg được thỏa mãn; nói cách khác, đối với mọi $\varepsilon>0$,
$$
a_n^{-1} \sum_{k=1}^n E\left[\left\|M_k-M_{k-1}\right\|^2 \mathbf{1}_{\left(\left\|M_k-M_{k-1}\right\| \geq \varepsilon a_n^{1 / 2}\right)} \mid \mathcal{F}_{k-1}\right] \xrightarrow{\mathrm{P}} 0 .$$
\end{itemize}
Khi đó
\begin{itemize}
    \item $a_n^{-1} M_n \xrightarrow{\text{a.s.}} 0$ và $a_n^{-1 / 2} M_n \xrightarrow{\mathcal{L}} \mathcal{N}(0, \Gamma)$. 
    \item  Nếu $\Gamma$ là khả nghịch thì: $a_n^{1 / 2}\langle M\rangle_n^{-1} M_n \xrightarrow{\mathcal{L}} \mathcal{N}\left(0, \Gamma^{-1}\right)$.
\end{itemize}
}
\section{Thuật toán Robbins-Monro và ước lượng đệ quy Nadaraya-Watson}
\subsection{Thuật toán Robbins-Monro}
{\dl (Thuật toán Robbins-Monro, tham khảo \cite{duflo}) Giả sử $\left(X_n\right)$ và $\left(Y_n\right)$ là hai dãy các vector ngẫu nhiên bình phương khả tích đáp ứng với bộ lọc $\mathbb{F}$ có giá trị trong $\mathbb{R}^d$ và $\left(\gamma_n\right)$ là một dãy các biến ngẫu nhiên dương giảm dần về không của các biến ngẫu nhiên tương thích với bộ lọc $\mathbb{F}$, với $\gamma_0 \leq$ hằng số, sao cho $X_{n+1}=X_n+\gamma_n Y_{n+1}$. Giả sử thêm rằng
$$
E\left[Y_{n+1} \mid \mathcal{F}_n\right]=f\left(X_n\right) \text {, và } E\left[\left\|Y_{n+1}-f\left(X_n\right)\right\|^2 \mid \mathcal{F}_n\right]=\sigma^2\left(X_n\right) .
$$
Khi đó, hàm $f$ là liên tục từ $\mathbb{R}^d$ vào $\mathbb{R}^d$ và bằng không tại điểm $x^*$ và đối với $x \neq x^*,\left\langle f(x), x-x^*\right\rangle<0$.
Chúng ta cũng giả định một trong các điều kiện sau:
\begin{enumerate}
    \item $d=1$ và $|f(x)| \leq K(1+|x|)$ với một hằng số $K$, và 
$$
\sum \gamma_n=\infty \text {, và } \sum \gamma_n^2 \sigma^2\left(X_n\right)<\infty \text{ h.c.c,}
$$
    \item $\sigma^2(x)+\|f(x)\|^2=s^2(x) \leq K\left(1+\|x\|^2\right)$ và,
$$
\sum \gamma_n=\infty \text { và } \sum \gamma_n^2<\infty \text { h.c.c .}
$$
\end{enumerate}
Khi đó $X_n \xrightarrow{\text { h.c.c }} x^*$.
}
%%%%%%%%%%%%%%%%%%%%%%%%%%%%%%%%%%%%%%%%%%%%%%%%%%%%%%%%%%%%%%%%%%
%%%%%%%%%%%%%% Hàm hạt nhân
%%%%%%%%%%%%%%%%%%%%%%%%%%%%%%%%%%%%%%%%%%5%%%%%%%%%%%%%%%%%%%%%%%
\subsection{Ước lượng đệ quy Nadaraya-Watson}
{\dn(Tham khảo \cite{noda}) Cho $\left(X_n\right)$ là dãy các biến ngẫu nhiên độc lập, cùng phân phối và hàm mật độ $f$ chưa biết.Cho $K$ là hàm đối xứng, dương, bị chặn và có giá compact sao cho
$$
\int_{\mathbb{R}} K(x) d x=1 \text { và } \int_{\mathbb{R}} K^2(x) d x=v^2
$$
thì ta gọi $K$ là hàm hạt nhân.
\noindent Ước lượng đệ quy Nadaraya-Watson của hàm $f$ được xác định bởi
$$
\hat{f}_n(x)=\frac{\sum_{k=1}^n W_k(x) Y_k}{\sum_{k=1}^n W_k(x)},
$$
trong đó
$$
W_k(x)=\frac{1}{h_k} K\left(\frac{X_k-x}{h_k}\right) .
$$
Băng tần $\left(h_n\right)$ là dãy các số thực dương, $h_n$ giảm dần về 0 và $n h_n$ tiến ra vô cùng. Với $0<\alpha<1$, ta thường dùng $h_n=1 / n^\alpha$.

{\dl (Tham khảo \cite{noda}) Với mọi $x \in \mathbb{R}$, nếu $\hat{f}_n(x)$ là ước lượng đệ quy Nadaraya-Watson của hàm $f$ thì
$$
\lim _{n \rightarrow \infty} \hat{f}_n(x)=f(x) \text { a.s. }
$$}
{\dl (Tham khảo \cite{schuster}) Giả sử rằng $\left(\varepsilon_n\right)$ có mô-men hữu hạn, bậc lớn hơn 2. Với mọi $x \in \mathbb{R}$, nếu $\frac{1}{5}<\alpha<1$ thì
$$
\sqrt{n h_n}\left(\hat{f}_n(x)-f(x)\right) \xrightarrow{\mathcal{L}} \mathcal{N}\left(0, \frac{\sigma^2 v^2}{(1+\alpha) g(x)}\right) .
$$
}
%%%%%%%%%%%%%%%%%%%%%%%%%%%%%%%%%%%%%%%%%%%%%%%%%%%%%%%%%%%%%%%%%%
%%%%%%%%%%%%%% Bổ đề Kronecker
%%%%%%%%%%%%%%%%%%%%%%%%%%%%%%%%%%%%%%%%%%5%%%%%%%%%%%%%%%%%%%%%%%
\section{Một số công cụ tính toán được sử dụng nhiều trong khóa luận}
{\bd (Bổ đề Kronecker, tham khảo \cite{duflo}) Giả sử $\left(a_n\right)$ là một dãy tăng ngặt các số thực dương tăng đến $\infty$ và $\left(x_n\right)$ là một dãy số thực sao cho chuỗi $\sum a_n^{-1} x_n$ hội tụ. Khi đó,
$$
a_n^{-1} \sum_{k=1}^n x_k \rightarrow 0 \text { khi } n \rightarrow \infty.
$$}
%%%%%%%%%%%%%%%%%%%%%%%%%%%%%%%%%%%%%%%%%%%%%%%%%%%%%%%%%%%%%%%%%%
%%%%%%%%%%%%%% 
%%%%%%%%%%%%%%%%%%%%%%%%%%%%%%%%%%%%%%%%%%5%%%%%%%%%%%%%%%%%%%%%%%

{\bd (Bổ đề Toeplitz, tham khảo \cite{duflo}) Cho $\left(a_n\right)$ là một dãy các số thực dương thỏa mãn
$$
\sum_{n=1}^{\infty} a_n=+\infty .
$$

Ngoài ra, cho $\left(x_n\right)$ là một dãy các số thực sao cho
$$
\lim _{n \rightarrow \infty} x_n=x .
$$

Khi đó, ta có
$$
\lim _{n \rightarrow \infty}\left(\sum_{k=1}^n a_k\right)^{-1} \sum_{k=1}^n a_k x_k=x .
$$}
%%%%%%%%%%%%%%%%%%%%%%%%%%%%%%%%%%%%%%%%%%%%%%%%%%%%%%%%%%%%%%%%%%
%%%%%%%%%%%%%% Dáng điệu tiệm cận
%%%%%%%%%%%%%%%%%%%%%%%%%%%%%%%%%%%%%%%%%%5%%%%%%%%%%%%%%%%%%%%%%%
{\dn Cho hai hàm số $f(x), g(x)$, ta nói hàm $f(x)$  là một "big-O" của $g(x)$, ký hiệu $f(x)=\mathcal{O}(g(x))$, khi $x$ dần về $a$
nếu tồn tại một hằng số $M,\delta$ sao cho $|f(x)| \leq M|g(x)|$ với mọi $x$ thỏa $|x-a|<\delta$}.
{\dn 
Ta viết
$$
f(x)=o(g(x)) \quad \text { khi } x \rightarrow \infty
$$
nếu với mọi hằng số dương $\varepsilon$, tồn tại một hằng số $x_0$ sao cho
$$
|f(x)| \leq \varepsilon g(x) \quad \text { đối với mọi } x \geq x_0 .
$$
Hay một cách tương đương là 
$$
\lim _{x \rightarrow \infty} \frac{f(x)}{g(x)}=0.
$$
{\dn 
Ta viết
$$
f(x)=\mathcal{O}(g(x)) \quad \text { khi } x \rightarrow \infty
$$
và đọc là "$f(x)$ là big-$\mathcal{O}$ của $g(x)$" nếu tồn tại số thực dương $M$ tự nhiên $x_0$ sao cho 
$$
|f(x)| \leq M g(x) \quad \text { với mọi } x \geq x_0 .
$$
}
{\md Một số tính chất cơ bản của "Big-$\mathcal{O}$" và "little-o"
\begin{enumerate}
    \item $f(x)=\mathcal{O}(f(x))$.
\item Nếu $f(x)=o(g(x))$ thì $f(x)=\mathcal{O}(g(x))$.
\item Nếu $f(x)=\mathcal{O}(g(x))$ thì $\mathcal{O}(f(x))+\mathcal{O}(g(x))=\mathcal{O}(g(x))$.
\item Nếu $f(x)=\mathcal{O}(g(x))$ thì $o(f(x))+o(g(x))=o(g(x))$.
\item Cho $c \neq 0$ thì $c \mathcal{O}(g(x))=\mathcal{O}(g(x))$ và $c o(g(x))=o(g(x))$.
\item $\mathcal{O}(f(x)) \mathcal{O}(g(x))=\mathcal{O}(f(x) g(x))$.
\item $o(f(x)) \mathcal{O}(g(x))=o(f(x) g(x))$.
\item Nếu $g(x)=o(1)$ thì $\frac{1}{1+o(g(x))}=1+o(g(x))$, và $\frac{1}{1+\mathcal{O}(g(x))}=1+\mathcal{O}(g(x))$.
\item Nếu $g(x)=\mathcal{O}(f(x))$ và $f(x)=o(h(x))$ thì $ g(x)=o(h(x))$.
\end{enumerate}
}




